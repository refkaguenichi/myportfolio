%%%%%%%%%%%%%%%%%
% This is an sample CV template created using altacv.cls
% (v1.6.4, 13 Nov 2021) written by LianTze Lim (liantze@gmail.com). Now compiles with pdfLaTeX, XeLaTeX and LuaLaTeX.
%
%% It may be distributed and/or modified under the
%% conditions of the LaTeX Project Public License, either version 1.3
%% of this license or (at your option) any later version.
%% The latest version of this license is in
%%    http://www.latex-project.org/lppl.txt
%% and version 1.3 or later is part of all distributions of LaTeX
%% version 2003/12/01 or later.
%%%%%%%%%%%%%%%%
%% If you need to pass whatever options to xcolor
\PassOptionsToPackage{dvipsnames}{xcolor}
%% Use the "normalphoto" option if you want a normal photo instead of cropped to a circle
% \documentclass[10pt,a4paper,normalphoto]{altacv}

\documentclass[10pt,a4paper,ragged2e,withhyper]{altacv}
%% AltaCV uses the fontawesome5 and packages.
%% See http://texdoc.net/pkg/fontawesome5 for full list of symbols.

% Change the page layout if you need to
\geometry{left=1.25cm,right=1.25cm,top=1.5cm,bottom=1.5cm,columnsep=1.2cm}

% The paracol package lets you typeset columns of text in parallel
\usepackage{paracol}

% Change the font if you want to, depending on whether
% you're using pdflatex or xelatex/lualatex
\ifxetexorluatex
  % If using xelatex or lualatex:
  \setmainfont{Roboto Slab}
  \setsansfont{Lato}
  \renewcommand{\familydefault}{\sfdefault}
\else
  % If using pdflatex:
  \usepackage[rm]{roboto}
  \usepackage[defaultsans]{lato}
  % \usepackage{sourcesanspro}
  \renewcommand{\familydefault}{\sfdefault}
\fi
% ----- DARK MODE -----
% \definecolor{BackgroundColor}{HTML}{242424}
% \definecolor{SlateGrey}{HTML}{6F6F6F}
% \definecolor{LightGrey}{HTML}{ABABAB}
% \definecolor{PrimaryColor}{HTML}{3F7FFF}
% \colorlet{name}{PrimaryColor}
% \colorlet{tagline}{PrimaryColor}
% \colorlet{heading}{PrimaryColor}
% \colorlet{headingrule}{PrimaryColor}
% \colorlet{subheading}{PrimaryColor}
% \colorlet{accent}{PrimaryColor}
% \colorlet{emphasis}{LightGrey}
% \colorlet{body}{LightGrey}
% \pagecolor{BackgroundColor}
% ----- LIGHT MODE -----
% Change the colours if you want to
% \definecolor{SlateGrey}{HTML}{2E2E2E}
% \definecolor{LightGrey}{HTML}{666666}
% \definecolor{DarkPastelRed}{HTML}{450808}
% \definecolor{PastelRed}{HTML}{8F0D0D}
% \definecolor{GoldenEarth}{HTML}{E7D192}
% \colorlet{name}{black}
% \colorlet{tagline}{PastelRed}
% \colorlet{heading}{DarkPastelRed}
% \colorlet{headingrule}{GoldenEarth}
% \colorlet{subheading}{PastelRed}
% \colorlet{accent}{PastelRed}
% \colorlet{emphasis}{SlateGrey}
% \colorlet{body}{LightGrey}
\definecolor{VividPurple}{HTML}{3E0097}
\definecolor{SlateGrey}{HTML}{2E2E2E}
\definecolor{LightGrey}{HTML}{666666}
\colorlet{heading}{VividPurple}
\colorlet{headingrule}{VividPurple}
\colorlet{accent}{VividPurple}
\colorlet{emphasis}{SlateGrey}
\colorlet{body}{LightGrey}
% Change some fonts, if necessary
\renewcommand{\namefont}{\Huge\rmfamily\bfseries}
\renewcommand{\personalinfofont}{\footnotesize}
\renewcommand{\cvsectionfont}{\LARGE\rmfamily\bfseries}
\renewcommand{\cvsubsectionfont}{\large\bfseries}


% Change the bullets for itemize and rating marker
% for \cvskill if you want to
\renewcommand{\itemmarker}{{\small\textbullet}}
\renewcommand{\ratingmarker}{\faCircle}

%% Use (and optionally edit if necessary) this .tex if you
%% want to use an author-year reference style like APA(6)
%% for your publication list
\input{pubs-authoryear}

%% Use (and optionally edit if necessary) this .tex if you
%% want an originally numerical reference style like IEEE
%% for your publication list
% \input{pubs-num}

%% sample.bib contains your publications
\addbibresource{sample.bib}

\begin{document}
\name{Refka GUENICHI}
\tagline{Fullstack Engineer}
%% You can add multiple photos on the left or right
% \photoR{2.8cm}{Globe_High}
% \photoL{2.5cm}{Yacht_High,Suitcase_High}

\personalinfo{%
  % Not all of these are required!
  \email{refkaguenichi@gmail.com}
  \linkedin{refka-guenichi-3a522a221}
  \github{refkaguenichi}
  \homepage{refka-guenichi-portfolio.netlify.app}
   \location{El Mourouj 6,Ben arous,2074, Tunisia}
    \phone{+21652957580}
%   \twitter{@twitterhandle}
 
 
%   \orcid{0000-0000-0000-0000}
  %% You can add your own arbitrary detail with
  %% \printinfo{symbol}{detail}[optional hyperlink prefix]
  % \printinfo{\faPaw}{Hey ho!}[https://example.com/]
  %% Or you can declare your own field with
  %% \NewInfoFiled{fieldname}{symbol}[optional hyperlink prefix] and use it:
  % \NewInfoField{gitlab}{\faGitlab}[https://gitlab.com/]
  % \gitlab{your_id}
  %%
  %% For services and platforms like Mastodon where there isn't a
  %% straightforward relation between the user ID/nickname and the hyperlink,
  %% you can use \printinfo directly e.g.
  % \printinfo{\faMastodon}{@username@instace}[https://instance.url/@username]
  %% But if you absolutely want to create new dedicated info fields for
  %% such platforms, then use \NewInfoField* with a star:
  % \NewInfoField*{mastodon}{\faMastodon}
  %% then you can use \mastodon, with TWO arguments where the 2nd argument is
  %% the full hyperlink.
  % \mastodon{@username@instance}{https://instance.url/@username}
}

\makecvheader
%% Depending on your tastes, you may want to make fonts of itemize environments slightly smaller
% \AtBeginEnvironment{itemize}{\small}

%% Set the left/right column width ratio to 6:4.
\columnratio{0.55}

% Start a 2-column paracol. Both the left and right columns will automatically
% break across pages if things get too long.
\begin{paracol}{2}
\cvsection{Experience}
\cvevent{Fullstack Engineer}{Podyam}{January 2022 -- Present}{Tunis}
\begin{itemize}
\item Develop and Design of "PodiumESG", it's a Fund Marketplace for Podyam's product "Podium360".\\
\cvtag{Vue}
\cvtag{Bootstrap}
\cvtag{Quasar}
\cvtag{Node}
\cvtag{Express}
\cvtag{Git}
\cvtag{PostgreSQL/Knex}
\cvtag{Docker}
\cvtag{Elasticsearch}
\cvtag{Azure}
\end{itemize}
\cvsection{Internships}
\cvevent{HVAC Engineering trainee}{K2A Engineering}{Mars 2019 --July 2019}{Tunis}
\begin{itemize}
\item Design of smart HVAC/Piping systems for buildings\\
\cvtag{IOT}
\cvtag{Arduino}
\cvtag{Photodiodes}
\cvtag{DHT22}
\cvtag{AutoCAD}
\end{itemize}

\cvevent{IOT Engineering trainee}{CRTEn}{June 2018 --August 2018}{Borj cedria}
\begin{itemize}
\item Design of smart tracking system\\
\cvtag{IOT}
\cvtag{Arduino}
\cvtag{Photodiodes}
\end{itemize}

\cvsection{Education}

\cvevent{Advanced Technologies Engineer Diploma}{National School of Sciences and Advanced Technologies in Borj Cedria ENSTAB}{2016 -- 2019}{}

\divider

\cvevent{Preparatory cycle Diploma in Maths&Physics}{IPEIK}{2014 -- 2016}{}

\divider

\cvevent{Baccalaureate Diploma in Maths}{Secondary of Djelma}{2010 -- 2014}{}
\medskip

\cvsection{Inerests}

% Adapted from @Jake's answer from http://tex.stackexchange.com/a/82729/226
% \wheelchart{outer radius}{inner radius}{
% comma-separated list of value/text width/color/detail}
\wheelchart{1.5cm}{0.5cm}{%
  6/8em/accent!30/Sports \\and travel,
  3/8em/accent!40/Artificial\\intelligence,
  8/8em/accent!60/{Dev web\\and mobile},
  2/10em/accent/IOT,
  5/6em/accent!20/Painting
}

% use ONLY \newpage if you want to force a page break for
% ONLY the current column
% \newpage
%% Switch to the right column. This will now automatically move to the second
%% page if the content is too long.

\switchcolumn

\cvsection{About me}

Passionate,self-motivated and team-worker full stack engineer. Excited to join your strong structure which is leader in the new technologies field and I look forward to this new challenge.

% ----- PROJECTS -----
        \cvsection{Own projects}
           \cvevent{findX - GoMyCode}{}{2021}{Tunis}
            \begin{itemize}
                 \item Lost and Found application, that helps users to tarck their objects.
                \item 
                \github{refkaguenichi/findx_app}
                \item 
                \homepage{findxapp.herokuapp.com/}
            \end{itemize}
            \cvtag{React}
            \cvtag{Node}
            \cvtag{Express}
            \cvtag{MongoDB}
            \cvtag{Material-ui}
            
            \divider
           \cvevent{swiftShop}{}{2021}{Tunis}
            \begin{itemize}
                \item E-commerce that facilitates online shopping, using Paypal as a payment method. 
                \item 
                \github{refkaguenichi/swift_shop}
                \item 
                \homepage{dropbox.com/s/zuuwjku24u3qlio/swiftshop.wmv}
            \end{itemize}
             \cvtag{React}
            \cvtag{Node}
            \cvtag{Express}
            \cvtag{MongoDB}
            \cvtag{Material-ui}
              \cvtag{Paypal}\\
               \smallskip
        % ----- PROJECTS -----

\cvsection{Skills}
            \cvtag{HTML}
            \cvtag{JS}
             \cvtag{TS}
            \cvtag{Python}
            \cvtag{PHP}
            \cvtag{CSS}
            \cvtag{SCSS}
            \cvtag{Bootstrap}
            \cvtag{Material-ui}
            \cvtag{Quasar}
            \cvtag{Vuetify}
            \cvtag{React}
            \cvtag{Vue}
            \cvtag{Node}
            \cvtag{Express}
            \cvtag{MongoDB}
            \cvtag{PostgresSQL}
            \cvtag{ObjectionJS}\\
            \cvtag{Knex}
            \cvtag{Sequelize}
            \cvtag{Docker}
            \cvtag{Git}
            \cvtag{Scrum}

\cvsection{Languages}

\cvskill{English}{4}
\divider
\cvskill{French}{4}
\divider
\cvskill{German}{1.5} 
\divider
\cvskill{Arabic}{5}


%% Yeah I didn't spend too much time making all the
%% spacing consistent... sorry. Use \smallskip, \medskip,
%% \bigskip, \vspace etc to make adjustments.
\medskip

% \NewInfoField{gitlab}{\faGitlab}[https://gitlab.com/]


\end{paracol}


\end{document}